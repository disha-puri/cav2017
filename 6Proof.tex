\section{Proof Sketch}
\label{sec:proof}

Certification of our loop pipelining algorithm naturally requires a certification of each of our primitives. In addition, we need to ensure that every time a primitive needs to be applied, the conditions under which the primitive can be applied are maintained. We discuss both aspects below.

\subsection{Correctness of Primitives}
We must prove that applying a particular primitive is correct, 
{\em i.e.}, maintaining a certain invariant. This is proven without
considering how it is applied in the context of a pipeline
synthesis algorithm. We give an outline of the proof to justify 
that the primitives are correct.

{\bf $\phi$-elimination primitive:} We prove that the algorithm correctly resolves the $\phi$ to create multiple assignment statements. We induct along the length of each sub-microstep of $\phi$-construct and
relate it to one corresponding assignment statement.

{\bf Shadow register primitive:} We prove that adding
  a shadow register microstep, $a\_reg = a'$ does not change the value of any variable in 
  the state except the shadow variable. In essence, we prove
  that if a variable is not written in a microstep, then its value in the state before and after executing
  that microstep is same. Also, we prove that after executing the shadow register microstep,
  value of $a\_reg$ in the state is equal to value of $a'$. Furthermore, since now
  the value of $a\_reg$ is equal to value
  of $a'$, we prove that executing a statement which reads $a'$ has the same
  effect on the state as executing a statement which reads
  $a\_reg$ till the next write of $a'$. This needs to be done for all types of
  statements {\em e.g.,} assignment statements (with different types of 
  operations like load, add, mul, getelementptr {\em e.t.c.}), 
  store statement, branch statement etc. We determine the variables read and 
  written in a statement by analyzing the statements. 
  Note that $a\_reg$ is a new variable which
  is neither written nor read in the given statements.

  {\bf Interchange primitive:} We prove that we can interchange any two adjacent microsteps
 (excluding branch microsteps) which do not have read-write conflict. Reasoning about read and
  write of statements involves reasoning about execution
  semantics of all types of microsteps present in the
  language which is not trivial.

{\bf Branch primitive:} We prove that executing $S$ $k$ times such that it exits 
in the $(k+1)$ st iteration is same as executing $S_{loop}$ $k$ times followed by $S_{preExit}$. 
We need to define a notion of a well-formed-flow to ensure that we can show that the 
branch does not exit in the first $k$ iterations. We also need to track the backedge along the 
unconditional branch and ensure that it points back to the beginning of loop $S$.
 
{\bf Superstep construction primitive:} This primitive is for overlapping iterations while 
maintaining data and control dependencies. It is built on interchange primitive but while interchange primitive
handles only two adjacent microsteps, superstep construction moves around scheduling steps with multiple microsteps. 
The interchange primitive is extended by non-trivial induction along the length of the scheduling steps to achieve the desired result. 
Superstep construction primitive is proved using the 
interchange primitive and a key invariant on correspondence between back-edges of sequential and pipelined loops~\cite{disha-itp14}. 

\subsection{Correctness of Our Algorithm}

Following is an English paraphrase of the final theorem.
  
\begin{quote}  
If the pipeline generation succeeds without error,
executing the pipelined CCDFG (a combination of $pre$, $loop$ and $post$) for $no\_pp\_steps$
generates the same state of the relevant variables as executing the sequential CCDFG $c$ for $no\_seq\_steps$.
%\small
%\begin{equation}
%\begin{split}
%no\_seq\_steps &= len S_{1} + (len S * (\ceil{\frac{m}{interval}} - 1)) + len S_{preExit} \\
%no\_pp\_steps &= len P_{pre} + (len P_{loop} * k) + len P_{post}
%\end{split}
%\end{equation}
\end{quote}

The algorithm is essentially built from ground-up using primitives
as shown in Section~\ref{sec:pipelining-algorithm}. However, 
apart from proving correctness of each primitive and our key 
invariant, we also need to ensure that the primitive is applied by 
our algorithm properly, {\em i.e.}, the environment
assumptions on which the {\bf correctness of primitive}
depends are maintained appropriately by the algorithm at
the point where the primitive is applied. 

We can take each stage one by one to understand the complexity involved in 
verifying the algorithm as a whole, over and above the verification of 
individual primitives. In the $Remove Branches$ stage, which is the first stage of pipelining algorithm, 
we have to create a correspondence between randomly executing a CCDFG with branches using basic-block, 
sub-basic-block and location with executing a CCDFG in sequence without a conditional and unconditional branch. 
After this step and for all the subsequent steps, we need to show that there are no relevant branches in CCDFG. 
In the $\phi$-to-assign stage, we replace one microstep of $C$ with more than one microsteps in $C'$. 
An inductive statement showing the correctness of $\phi$-elimination must account for the fact
that the number of microsteps of $C$ is different from that
of $C'$.  Thus an execution of $C$ for $n$ microsteps must
correspond to an execution of $C'$ for a different number
$m$ of microsteps, where the number $m$ is a function of $n$
and the structures of $C$ and $C'$; the statement of the
correctness of $\phi$-elimination must characterize the
value of $m$ precisely, perhaps defining functions that
statically and symbolically execute $C$ and $C'$, in order
to be provable by induction.  Furthermore the functions so
introduced for static symbolic execution must themselves be
proven correct. $Data propogation$ involves identifying 
the appropriate statements that cause conflict and applying 
interchange primitive multiple times 
to move the microstep to the beginning of the loop. 
We need to make sure that the conditions under which
interchange primitive can be further applied are maintained after each application. 
$Shadow register step$ also adds many more new statements to assign temporary values to 
new shadow variables. The addition of new microsteps means that in addition to 
inductively reasoning about application of a primitive in entire CCDFG, we also have to 
ensure that basic structure of the CCDFG is maintained. Moreover, we need to reason about read and write of 
variables across a number of microsteps. The proof is analogous to the proof of shadow-register primitive. 
However, the primitive is applied multiple times based on the variabes which are causing conflict. 
This gets tricky as after application of one primitive, there are new variables introduced and 
we can only claim that the relevant variables have same value.This step requies proof of invariant and multiple applications of interchange primitive as explained earlier. The proof required is the reverse of branch-primitive. However, a key requirement is that branch-primitive can be applied only when we have a {\tt well-formed-ccdfg}, so we need to ensure that the structure of the $loop$ before adding 
branches is such that the final $loop$ in the pipelined CCDFG is indeed a {\tt well-formed-ccdfg}.  

