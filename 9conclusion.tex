\section{Conclusion and Future Work}
\label{sec:research-plan}

We have developed a framework of succinct certified primitives essential to build pipelining algorithms. We utilize our framework of certified primitives as backbone to build our certified loop pipelining algorithm. Since, we have a certified loop pipelining algorithm, we can confidently say that there are no data hazards and executing a sequential loop is same as executing a pipelined loop created using our algorithm. We have tested the pipeline reference model created using our algorithm on a variety of ESL designs across different application domains. Table 2 gives a high level overview of some of the ACL2 effort involved in such kind of work. %This shows that our algorithm is practical and can be used for industrial strength designs with tens of thousands of RTL.  Our current ACL2 script has $296$ definitions and $1012$ lemmas, including many lemmas about structural properties of CCDFGs (but not counting those from the false starts). 

\begin{table}
  \centering
  \label{fig:proof}
  \begin{tabular}{| c || c | c |}
    \hline
    & Definitions & Lemmas \\
    \hline
    Overall Certification & 296 & 1012 \\
    \hline
    CCDFG Formalization & 107 &  \\
    \hline
    Removing Branches & 26 & 286 \\ 
    \hline
    Data Propagation & 19 & 59 \\ 
    \hline
    Shadow Register & 46 & 143 \\
    \hline
    Correspondence Invariant & 18 & 122 \\ 
    \hline
  \end{tabular}
        \caption{ACL2 Effort}
\end{table}

%Our work shows that it is possible to develop and certify an industrial-strength loop pipelining algorithm if we can decompose it into succint certifiable primitives. We have already identified and certified these primitives. 
Our algorithm has components which can identify data hazards based on the given pipeline interval. Then we use our certified primitives to remove those data hazards and create a pipelined implementation. Function pipelining algorithms also have the same type of data hazards as we have mentioned in loop pipelining algorithms. However, while loop pipelines have a fixed pipeline interval which is known at compile time, function pipelines have a variable pipeline interval for every iteration. So, instead of identifying data hazards at once for every iteration, we would have to call those functions for each iteration. After we have identified the data hazards, we can use our certified primitives to remove those data hazards. We believe that if we can modify the algorithm to identify data hazards, then we can conveniently reuse our certified primitives to certify behaviorally synthesized function pipelines as well.     