\section{Introduction}
\label{sec:intro}

In recent years, the demand for hardware with smaller form factor and higher transistor density has been steadily increasing. Consequently, it has become difficult to create high quality hardware by hand-crafting RTL designs. A behavioral synthesis tool takes a behavioral
design description (in C, C++, or SystemC) and automatically generates an optimized RTL design. 
%Studies have shown that ESL 
%reduces the design effort by $50\%$ or more while attaining
%excellent performance results~\cite{Moussa99}. 
It has recently received significant attention, as the steady 
increase in hardware complexity has made it increasingly 
difficult to design high-quality designs through hand-crafted RTL under 
aggressive time-to-market schedules~\cite{googledecoder}. Nevertheless, and in spite of availability of several
commercial behavioral synthesis tools~\cite{ctos,forte,vivado},
the adoption of the approach in main-stream hardware development for
microprocessor and SoC design companies is dependent on 
designers' confidence that the synthesized RTL indeed corresponds to the ESL
specification. 

Loop pipelining is one of the most complex transformations in behavioral synthesis. It is available in most commercial synthesis tools~\cite{autoesl,forte}
% ({\em e.g.}, AutoESL~\cite{autoesl} and Cynthesizer~\cite{forte}) 
and is
crucial to producing hardware with high throughput and low latency by allowing temporal overlap of successive loop
iterations. Certifying the correspondence between a sequential design and its pipelined counterpart is challenging due to the huge semantic gap between the two
designs. As a result, hardware designers are vary of using current behavioral synthesis tools as
they are often deemed either (a) aggressively optimized but error-prone or (b) 
reliable but overly conservative, thus often producing circuits of poor quality
or performance~\cite{spark,kundu2008}. Therefore, ensuring
correctness of behaviorally synthesized pipeline designs
is a critical issue in bringing behavioral synthesis into practice.

An approach for certifying loop pipelining transformations using a combination of SEC and 
theorem proving techniques has been proposed earlier~\cite{hrx:dac-12}. The most critical and complex
component of this approach is developing
a loop pipelining algorithm with two key properties: (1) it generates a reference pipeline model 
by exploiting pipeline generation information from the synthesis flow (e.g., the iteration interval 
of a generated pipeline) and the reference model can be compared with a pipelined RTL 
implementation using SEC effectively, and (2) it can be mechanically verified to correctly preserve the semantics of
sequential (non-pipelined) specification of loop execution. The viability of 
this approach was shown by comparing pipeline generated from this algorithm with RTL implementation 
using SEC. However, this algorithm was not certified as well as incomplete. 
In fact, during our certification process we found a number of errors in the algorithm, 
which makes the entire certification flow for behavioral synthesis suspect. 
%Certification is a key component without which correctness of behavioral synthesis process for pipelined designs cannot be claimed.
Our work on developing a certified loop pipelining algorithm using our framework of certified 
primitives is important to facilitating formal verification of behaviorally synthesized
 pipeline designs.

\medskip
The key contributions of this paper are:
\begin{enumerate}
\item Developing a certified executable loop pipelining algorithm in ACL2 using our framework of certified pipelining primitives 
\item Demonstrating the viability of the certified loop pipelining algorithm in a behavioral synthesis certification flow for industrial hardware designs
\end{enumerate}

The remainder of this paper is organized as
follows. Section~\ref{sec:background} provides background on the need for a certified loop pipelining algorithm 
to certify behaviorally synthesized designs. Section~\ref{sec:formalization} 
discusses our formalization of the intermediate representations used in behavioral synthesis.
We also discuss the correctness statement for loop pipelining algorithms. 
We discuss our framework and a certified loop pipelining algorithm we have developed using the framework in Section~\ref{sec:pipelining-algorithm}. Section~\ref{sec:proof} provides a proof sketch for our algorithm and Section~\ref{sec:SEC} provides evaluation of robustness and 
scalability of our algorithm on industrial-strength designs. Section~\ref{sec:related-work} discusses the work that has been done in the area of microprocessor and software pipeline verification before and how our work is different. We then conclude and discuss future work in Section~\ref{sec:research-plan}.


